% #!latex tesi.tex
\chapter{Il problema dell'etere}
Il processo che port\`o alla relativit\`a ristretta fu lungo e per
nulla lineare. Uno dei punti chiave che permise di arrivarvi fu
l'abbandono dell'idea dell'etere, il mezzo nel quale le onde (luminose
ed elettromagnetiche) dovevano vibrare per potersi propagare. Si
ipotizz\`o l'esistenza di due tipi di etere: quello luminifero, e
quello elettromagnetico.

Per una prima svolta si dovette aspettare il 1856, quando Weber e
Kohlrausch misurarono la velocit\`a delle onde elettromagnetiche,
ossia
\begin{equation}
  c = \frac{1}{\sqrt{\varepsilon_0 \mu_0}} = 3 \cdot 10^8 m / s,
\end{equation}
uguale alla velocit\`a della luce nel vuoto: onde elettromagnetiche e
luce dovevano esser parte dello stesso fenomeno; da ci\`o si dedusse
l'esistenza di un solo etere, sul quale tuttavia non si sapeva niente
se non che esso era soggetto alle equazioni di Maxwell (il quale
elabor\`o una complicata quanto infruttuosa teoria dello stesso per
mezzo di modelli meccanici).

Seguirono numerose modellizzazioni, che per\`o non si affermarono mai
e per il loro grado di complicazione e perch\'e non avevano capacit\`a
di previsione; si dovr\`a aspettare Heinrich Hertz, il quale
abbandon\`o la pretesa di spiegare l'etere in s\'e stesso, per
dedicarsi al solo studio dei fenomeni ad esso correlati: per quanto
riguarda il suo stato, si interess\`o solo alle due grandezze vettoriali
che lo descrivevano: il campo elettrico $\bef{E}$, ed il campo magnetico
$\bef{B}$.

Con l'avvento dell'elettromagnetismo di Maxwell tuttavia, non solo era
stato dapprima necessario introdurre un nuovo tipo di etere, ma era
sorto anche il seguente problema: le sue equazioni non erano
covarianti per le trasformazioni di Galileo. Il problema poteva
avere principalmente due soluzioni:
\begin{enumerate}
\item le equazioni di Maxwell sono valide solo in un particolare
  sistema di riferimento (e subito si pens\`o all'etere), mentre hanno
  forma simile per gli altri
  sistemi di riferimento.

  Tuttavia un esperimento, ad opera di Michelson e Morley, minava alle
  fondamenta quest'ipotesi, dacch\'e non rilevava alcun moto relativo
  (fino al secondo ordine di $v / c$) della terra rispetto all'etere: la
  velocit\`a della luce (che doveva avere modulo pari a $1/(\mu_0
      \varepsilon_0)^{0.5}$ nei sistemi di
  riferimento dove valevano le equazioni di Maxwell\footnote{E quindi
      non sulla terra, ma solo sull'etere; essa risultava un sistema di
      riferimento diverso in virt\`u del fenomeno dell'aberrazione della
      luce.}) non subiva alcuna influenza dal moto terrestre.

  Analizzando le equazioni per i tempi di propagazione coinvolti
  nell'esperimento, questa risultava avere due spiegazioni:
  \begin{enumerate}
  \item l'etere \`e completamente trascinato dal moto della terra.
  \item ogni corpo in movimento contrae il proprio lato parallelo alla
    velocit\`a di un fattore $\sqrt{(1-v^{2}/c^{2})}$.
  \end{enumerate}
  La prima ipotesi fu avanzata da Stokes, ma si rivel\`o inaccettabile
  poich\'e portava, nei suoi sviluppi formali, a contraddizioni
  inaccetabili.%  
  \newline% 
  La seconda ipotesi fu invece avanzata da
  FitzGerald~\cite{fitz}, un fisico irlandese, nel 1889, e
  incredibilmente, si rivel\`o la pi\`u fondata, bench\'e fosse solo un
  primo passo per la soluzione del problema;

\item le equazioni di Maxwell sono covarianti: non per le
  trasformazioni di Galileo per\`o, ma per altre, trovate da Lorentz, 
  il quale riprese (ignorando il lavoro di Fitzgerald) nel 1895 la
contrazione delle lunghezze, e propose anche la dilatazione dei tempi
(vedi~\cite{loris1895}), poich\'e in questo modo era possibile non solo
spiegare il fallimento dell'esperimento ma anche introdurre delle
trasformazioni (denominate, da \poin, di Lorentz) sotto le quali era
assicurata la covarianza delle equazioni di Maxwell.
\end{enumerate}

Le trasformazioni proposte per coordinate spaziali e temporali tra
$(x,y,z,t)$ e $ (x',y',z',t')$, in moto con velocit\`a relativa $v$
sono\footnote{La forma in cui le riportiamo non \`e quella proposta da
  Lorentz, ma da \poin{} nel 1905.}
\begin{equation}
  \left\{
    \begin{array}{rcl}
      x' & = & \gamma (x - v t)  \\
      y' & = & y \\
      z' & = & z \\
      t' & = & \gamma (t - v x / c^{2})
    \end{array}
  \right.
  \label{eq:tloris}
\end{equation}
dove 
\begin{equation}
     \gamma = \frac{1}{\sqrt{1 - v^{2}/c^{2}}}.
\end{equation}

\begin{observazione}
  Facciamo notare che le trasformazioni da noi scritte presentano la
  costante $c$, ossia la velocit\`a della luce nel vuoto. Il perch\'e
  $c$ comparisse era legato al fatto che in tal modo le cose
  funzionavano: non c'era nessuna assunzione teorica alla base che
 potesse giustificarlo.

  Lorentz, d'altronde, necessitava di quest'assunzione, poich\'e nel
  1892, in~\cite{loris1892}, aveva affermato che
  \begin{quotation}
    L'etere \`e a riposo nello spazio assoluto,
  \end{quotation}
  e questo implicava che esso fosse un sistema di riferimento
  assoluto, in cui valevano le equazioni di Maxwell: da qui la
  necessit\`a dell'assunzione precedente.
\end{observazione}

Sorge ora spontanea una domanda: perch\'e se due osservatori in
sistemi di riferimento diversi possono affermare di trovarsi entrambi
fermi nell'etere (dacch\'e in virt\`u delle trasformazioni di Lorentz
le leggi dell'elettromagnetismo sarebbero state le stesse rispetto al
sistema ``assoluto''), mentre entrambi si muovono rispetto ad esso,
\`e ancora necessario questo concetto di ``etere''?  \newline
Probabilmente la risposta \`e insita nel fatto che non si riusciva ad
immaginare come un'onda riuscisse a propagarsi in assenza di un mezzo
in cui vibrare: quindi Lorentz, e molti altri con lui, continu\`o a
credere in questo sistema ``introvabile'' (poich\'e da nessun
esperimento si sarebbe riusciti a capire la propria presenza o meno in
esso).
