%#!latex tesi.tex
\chapter{\poin}
\minitoc
Nella sua principale memoria sulla relativit\`a~\cite{carro1}, che
riassume alcune sue considerazioni sull'argomento, Jules Henry \poin {}
(1854-1912) viene motivato dall'inconsistenza dei risultati
dell'esperimento di Michelson e Morley con la teoria fino ad allora
sviluppata.
%
%Infatti
%leggiamo in~\cite{carro1}:
%\begin{citaz}
%  A prima vista sembra che l'aberrazione della luce indichi che la
%  terra sia in moto assoluto rispetto all'etere. Ma tutti gli
%  esperimenti, compreso quello di Michelson e Morley, sembrano non
%  rilevare tale moto.
%\end{citaz}
Parte dunque da un risultato fisico, e cerca di darne una spiegazione;
%continua infatti nella sua memoria esponendo il
nella memoria espone il
\begin{principio}[di Relativit\`a di \poin\footnote{Confronta
    \cite{carro2}}]
  Le leggi dei fenomeni fisici sono le stesse per tutti i sistemi di
  riferimento inerziali.  Dunque non esiste mezzo alcuno per stabilire
  se siamo in moto o meno\footnote{Tuttavia non \`e la prima volta che
    \poin{} lo espone. Gi\`a in~\cite{carro4} possiamo trovare un
    intero capitolo dedicato al principio di relativit\`a, sebbene non
    sia espresso in questa forma.}.
\end{principio}
Il suo obiettivo, nella prima parte della memoria, quella interessante
per i nostri scopi, \`e dimostrare che tramite le trasformazioni di
Lorentz, da lui emendate, le leggi dell'elettromagnetismo rimangono
invariate. Egli stesso dice
\begin{citaz}
  Queste equazioni [di Maxwell] rimangono invariate se vi si applicano
  le trasformazioni di Lorentz, e questo spiega perch\'e nessun
  esperimento \`e in grado di rilevare il moto della terra rispetto
  all'etere.
\end{citaz}
\section{Le trasformazioni di Lorentz}
Poincar\'e nel suo lavoro non deriva direttamente le trasformazioni di
Lorentz, bens\`i dice di trovarle (e gli si pu\`o credere, visto che le
scrive in maniera corretta, al contrario di Lorentz). Mostriamo come, in
effetti, dall'assunzione dell'omogeneit\`a ed isotropia dello
spazio, assieme al principio di relativit\`a, e alla richiesta, motivata
dall'interpretazione fisica, che costituiscano un gruppo, \`e possibile arrivare
alle trasformazioni nel seguente modo.

 Considereremo le sole coordinate $(x,t)$ e $(x',t')$,
 come in figura~\vref{fig:traslazione1}.
 \setlength{\unitlength}{0.7mm}
 \begin{figure}[h]
   \begin{center}
     \begin{picture}(80,15) \put(6,1){$O$} \put(13,2){\vector(1,0){40}}
       \put(58,1){$x$} \put(0,9){$O'$} \put(7,10){\vector(1,0){40}}
       \put(52,9){$x'$} \put(65,7){$\frac{\de \overline{OO'}}{\de t} =
         v$}
     \end{picture}
   \end{center}
   \caption{Il sistema $O'x'$ \`e in moto relativo rispetto al sistema
     $Ox$, con velocit\`a $v$.}
   \label{fig:traslazione1}
 \end{figure}
 In questo caso ci interesseremo delle trasformazioni $(x,t)
 \rightarrow (x',t')$, che chiamiamo trasformazioni
 $\mathcal{A}$. Girando il sistema $O'x'$ di $\pi$, otterremmo la
 configurazione mostrata in figura~\vref{fig:traslazione2}.
 \begin{figure}[tb]
   \begin{center}
     \begin{picture}(80,15) \put(6,1){$O$} \put(13,2){\vector(1,0){40}}
       \put(58,1){$x$} \put(0,9){$x'$} \put(47,10){\vector(-1,0){40}}
       \put(52,9){$O'$} \put(65,7){$\frac{\de \overline{OO'}}{\de t} =
         -v$}
     \end{picture}
   \end{center}
   \caption{Il sistema $O'x'$ \`e in moto relativo rispetto al sistema
     $Ox$, con velocit\`a $-v$.}
   \label{fig:traslazione2}
 \end{figure}
 In questo caso prenderemmo in esame le trasformazioni $(x',t')
 \rightarrow (x,t)$; chiamiamole trasformazioni $\mathcal{B}$.

 L'omogeneit\`a dello spazio e del tempo comporta che $\mathcal{A}$ e
 $\mathcal{B}$ siano lineari. Poniamo $t=t'=0$ quando $x=x'=0$ per
 entrambe le trasformazioni, che dunque risultano ($A, \ldots, D'$
 costanti)
 \begin{equation}
   \mathcal{A} \left\{
     \begin{array}{rcl}
       x' & = & A x + B t \\
       t' & = & Cx + Dt
     \end{array}
   \right.	
   \quad
   \mathcal{B} \left\{
     \begin{array}{rcl}
       x & = & A' x' + B' t' \\
       t & = & C' x' + D' t'.
     \end{array}
   \right.
   \label{eq:loris}
 \end{equation}
 Per il principio di relativit\`a, e per la simmetria del problema, si
 riconosce che deve essere $A=A', \ldots, D=D'$. Inoltre se per
 $\mathcal{A}$ si prende $x' = 0$ si ha, ricordando che $x = v t$, che
 $A v + B = 0$. Per la seconda delle $\mathcal{B}$ invece prendiamo
 sempre l'origine di $O' x'$, e dunque, sostituendo $t = D t'$ nella
 prima di $\mathcal{B}$, si ottiene (sempre ricordando che $x' = 0$) $D
 v = B$, da cui $A = -D$. Ora queste relazioni devono valere $\forall
 \, x',\,t'$, in quanto $A, \ldots, D$ sono costanti. Posso dunque
 operare una sostituzione nella prima di $\mathcal{B}$, per ottenere
 \begin{eqnarray}
   x & = & A x' + B t' \nonumber \\ 
   & = & A (Ax + Bt) + B (Cx + Dt) \nonumber \\
   & = & t (AB + BD) + x (A^{2} + BC) \nonumber \\
   & = & x (D^{2} + CDv)
 \end{eqnarray}
 \begin{equation}
   D^{2} + CDV  =  1 \Longrightarrow
   C  =  \frac{1-D^{2}}{DV}.
 \end{equation}
 Sostituendo i valori ottenuti per $A, \, B$ e $C$ nella $\mathcal{A}$,
 si ottiene
 \begin{equation}
   \left\{
     \begin{array}{rcl}
       x' & = & -D x + D vt \\
       t' & = & \frac{1-D^{2}}{D v} x + D t.
     \end{array}
   \right.
   \label{eq:loris1}
 \end{equation}
A questo punto, per trovare i rimanenti parametri incogniti, basta
comporre due trasformazioni; si giunge alle trasformazioni (i sistemi
sono posti come in figura~\ref{fig:traslazione1})\footnote{Tali
trasformazioni presentano il parametro $c$, imposto pari alla velocit\`a
della luce per il fatto che, in tal modo, le cose ``funzionavano''.}
\begin{equation}
  \left\{
    \begin{array}{rcl}
      x' & = & l \gamma (x - v t)  \\
      y' & = & l y \\
      z' & = & l z \\
      t'  & = & l \gamma (t - v x / c^{2}),
    \end{array}
  \right.
  \label{eq:lorentzelle}
\end{equation}
dove $l = l (\beta)$: per trovarlo imponiamo che le trasformazioni di
Lorentz formino un gruppo: prendiamo la trasformazione tra due sistemi
come in figura~\ref{fig:traslazione2}: essa avr\`a la forma 
\begin{equation}
  \left\{
    \begin{array}{rcl}
      x' & = & l \gamma (x + v t)  \\
      y' & = & l y \\
      z' & = & l z \\
      t'  & = & l \gamma (t + v x / c^{2});
    \end{array}
  \right.
  \label{eq:lorentzvmeno}
\end{equation}
d'altronde, se esse formano un gruppo, dovr\`a valere l'inversa
della~\eqref{eq:lorentzelle}, ossia
\begin{equation}
  \left\{
    \begin{array}{rcl}
      x' & = &  \gamma (x + v t)/l  \\
      y' & = & y/l \\
      z' & = &  z/l \\
      t'  & = & \gamma (t + v x / c^{2}) /l.
    \end{array}
  \right.
  \label{eq:lorentz_inv}
\end{equation}
Allora $l=1/l$, da cui $l=\pm 1$: essendo per\`o $l=1$ per $v=0$, si
avr\`a sempre $l=1$: da ci\`o le famose trasformazioni.
%A questo punto per trovare le trasformazioni di Lorentz, basta far
%notare che esse devono costituire un gruppo: infatti l'unico parametro
%sconosciuto \`e $D$. Per trovarlo consideriamo il sistema in
% figura~\vref{fig:traslazione3};
% \begin{figure}[h]
%   \begin{center}
%     \begin{picture}(130,25) \put(6,1){$O$}
%       \put(13,2){\vector(1,0){40}} \put(58,1){$x$} \put(3,9){$O'$}
%       \put(10,10){\vector(1,0){40}} \put(55,9){$x'$} \put(0,18){$O''$}
%       \put(7,19){\vector(1,0){40}} \put(52,18){$x''$}
%       \put(65,7){$\frac{\de \overline{OO'}}{\de t} = v$}
%       \put(95,7){$\frac{\de \overline{OO''}}{\de t} = v''$}
%       \put(125,7){$\frac{\de \overline{O'O''}}{\de t} = v'$}
%     \end{picture}
%   \end{center}
%   \caption{Il sistema $O'x'$ \`e in moto relativo rispetto al sistema
%     $Ox$, con velocit\`a $v$, il sistema $O''x''$ \`e in moto relativo
%     rispetto al sistema $Ox$ con velocit\`a $v''$, e con velocit\`a
%     $v'$ rispetto a $O'x'$.}  \label{fig:traslazione3}
% \end{figure}
% la~\eqref{eq:loris1} diventa allora (per l'inversione delle $x'$)
% \begin{equation}
%   \left\{
%     \begin{array}{rcl}
%       x' & = & D (x - vt) \\
%       t' & = & \frac{1-D^{2}}{D v} x + D t.
%     \end{array}
%   \right.
%   \label{eq:loris2}
% \end{equation}
% Scriviamo ora le trasformazioni per $O''x''$ rispetto a $O'x'$ e $Ox$,
% ottenendo ($D'$ non \`e quello di~\eqref{eq:loris})
% \begin{equation}
%   \mathcal{C}
%   \left\{
%     \begin{array}{rcl}
%       x'' & = & D' (x' - v't') \\
%       t'' & = & \frac{1-D'^{2}}{D' v'} x' + D' t'.
%     \end{array}
%   \right.
%   \quad
%   \mathcal{D}
%   \left\{
%     \begin{array}{rcl}
%       x'' & = & D'' (x - v''t) \\
%       t'' & = & \frac{1-D''^{2}}{D' v''} x + D'' t.
%     \end{array}
%   \right.
%   \label{eq:loris3}
% \end{equation}
% Usando la~\eqref{eq:loris2} in $\mathcal{C}$, otteniamo
% \begin{equation}
%   \mathcal{D'}
%   \left\{
%     \begin{array}{rcl}
%       x'' & = & \left[ DD' + D' v' \frac{D^{2} -1}{D v} \right] x -DD' (V+V') t \\
%       t''  & = & \left( \frac{D - DD'^{2}}{D' v'} +
%         \frac{D' - D'D^{2}}{D v}\right) x
%       + \left( DD' + Dv \frac{D'^{2} -1}{D'V'}\right) t.
%     \end{array}
%   \right.
%   \label{eq:loris4}
% \end{equation}
% Ovviamente $\mathcal{D}$ e $\mathcal{D'}$ devono essere uguali,
% perci\`o, tra le altre cose, devono valere
% $$ \left\{
% 	\begin{array}{l}
%           D''  =  DD' + D'v' \frac{D^{2} - 1}{D v}\\
%           D''  =  DD' + Dv \frac{D'^{2} - 1}{D' v}
% 	\end{array}
%       \right.  \Rightarrow D'' - D = D' v' \frac{D^{2} - 1}{D v} = Dv
%       \frac{D'^{2} - 1}{D' v}. \nonumber
% $$
% Quest'ultima equazione ci permette di definire la costante $K$,
% indipendente da $v$
% \begin{equation}
%   K = \frac{D^{2} v^{2}}{D^{2}-1} = \frac{D'^{2} v'^{2}}{D'^{2}-1}.
%   \label{eq:kappa}
% \end{equation} 
% Possiamo ora determinare $D$ in funzione di $K$;
% dalla~\eqref{eq:kappa} avremmo due soluzioni, ma dacch\'e per $v = 0$,
% $D = 1$ (l'equazione~\eqref{eq:loris2} \`e l'identit\`a), allora, in
% generale, dovremo scegliere la soluzione positiva:
% \begin{equation}
%   D = \frac{1}{\sqrt{1 - v^{2}/K^{2}}}.
%   \label{eq:di}
% \end{equation}
%Con questo metodo del tutto generale possiamo ottenere sia le
%trasformazioni di Galileo ($K = \infty$), sia le trasformazioni
%``corrette'' ($K = c^2$). Tuttavia n\'e Lorentz n\'e Poincar\'e
%avevano sviluppato una teoria dalla quale uscisse in maniera naturale
%$K = c^2$: la loro scelta fu dettata dalla necessit\`a, in quanto in
%questa maniera si spiegava l'esperimento di Michelson e Morley, e in
%particolare \poin {} dimostr\`o che, in tal modo, le equazioni di
%Maxwell rimangono invariate passando da un sistema di riferimento
%inerziale all'altro (in accordo con il principio di
%relativit\`a). Vediamo come.

\section{Le equazioni di Maxwell}
In questa sezione, volendo riportare il pi\`u fedelmente possibile
l'articolo~\cite{carro1}, useremo la notazione di Gauss razionalizzata,
e le unit\`a spaziali e temporali in maniera tale da avere $c =
1$. Inoltre per le trasformazioni di Lorentz useremo un parametro $l = l
(\beta)$, che, come abbiamo visto, \poin {} porr\`a pari a 1 per
garantire le propriet\`a gruppali delle trasformazioni\footnote{\poin{}
prima affronta questa sezione, e poi dimostra che $l=1$.
}.

Le equazioni che ci interessano sono ($\bef{u}$ \`e la velocit\`a)
\begin{equation}
  \partial \bef{E} / \partial t + \rho \bef{u} = \nabla \times \bef{B}
  \label{eq:corrente}
\end{equation}
\begin{equation}
  \bef{B} = \nabla \times \bef{A}, \quad \bef{E} = 
  - \partial \bef{A} / \partial t - \nabla \varphi
  \label{eq:aephi}
\end{equation}
\begin{equation}
  \partial \bef{B} / \partial t = - \nabla \times \bef{E}
  \label{eq:bede}
\end{equation}
\begin{equation}
  \partial \rho / \partial t + \nabla \cdot \rho \bef{u} = 0
  \label{eq:rhoeu}
\end{equation}
\begin{equation}
  \nabla \cdot \bef{E} = \rho
  \label{eq:rhoede}
\end{equation}
\begin{equation}
  \partial \varphi / \partial t + \nabla \cdot \bef{A} = 0
  \label{eq:phiea}
\end{equation}
\begin{equation}
  \Box \varphi = - \rho, \quad \Box \bef{A} = - \rho \bef{u}
  \label{eq:phiaerho}
\end{equation}
dove $\Box = \nabla^{2} - \partial^{2} / \partial t^{2}$. Inoltre egli
prende in considerazione la forza e la densit\`a di forza
elettromagnetica:
\begin{equation}
  \label{eq:forza}
  \bef{F} = (\bef{E} + \bef{v} \times \bef{B})
\end{equation}
\begin{equation}
  \label{eq:denforza}
  \bef{f} = \rho (\bef{E} + \bef{v} \times \bef{B}).
\end{equation}

Cominciamo la nostra trattazione con la legge di trasformazione dei
volumi. Prendiamo una sfera che si muove, nel sistema di riferimento
$(x,y,z,t)$ con velocit\`a $\bef{u}$. Il suo volume \`e $4/3 \pi
r^{3}$. Nel sistema di riferimento $(x',y',z',t')$, solidale alla
sfera, la sua equazione \`e (usando le trasformazioni) per $t'=0$ (e
trovato il volume a questo tempo, esso rimane costante)
\begin{displaymath}
  \gamma^{2} \left( x_{1}' - \beta u_{1} x_{1}' \right)^{2} + 
  \sum_{j=2}^{3}\left(x_{j} - \gamma u_{j} \beta x_{1}'\right)^{2} =
 l^{2} r^{2}. 
\end{displaymath}
Con un'integrazione possiamo trovare il volume, che risulta $V' = V
l^{3} [\gamma(1-\beta u_{1})]$. Dunque $V' / V = l^{3} / [\gamma
(1-\beta u_{1})]$, da cui, considerando (come assume \poin) che la carica elettrica non
cambia se cambiamo il sistema di riferimento, e che perci\`o $\rho V =
\rho' V'$, si ha
\begin{equation}
  \rho' = \rho l^{-3} \gamma (1 - \beta u_{1}).
  \label{eq:rhoprimo}
\end{equation}
Enunciamo ora la regola di composizione delle velocit\`a, diretta
conseguenza delle trasformazioni di Lorentz
\begin{equation}
  u_{1}' = (u_{1} - \beta) / (1 - \beta u_{1}), 
  \quad  
  u_{j}' = u_{j} / \gamma (1-\beta u_{1}) 
  \quad 
  (j=2,3).
  \label{eq:vel}
\end{equation}
Questo ci permette di scrivere
\begin{displaymath}
  \rho' u_{1}' = \gamma l^{-3} (u_{1} - \beta) \rho,
  \quad
  \rho' u_{j}'  = l^{-3} \rho u_{j},
  \quad
  (j=2,3).
\end{displaymath}
Tali formule ci torneranno utili pi\`u tardi. Cerchiamo ora di capire
come trasforma $\Box$. Per comodit\`a indichiamo con $x^{0}$ il tempo
$t$, e $x^{i}$ le coordinate $x \,(i=1)$, $y \,(i=2)$ e $z \,
(i=3)$. Le corrispondenti quantit\`a con gli apici saranno indicate
con le componenti di $y$\footnote{Nella~\eqref{eq:lorentzelle} vi sono
  le $y$ scritte in funzione delle $x$ }.  Da ci\`o

\begin{eqnarray*}
  \Box & = & \sum_{1}^{3} \frac{\partial}{\partial x^{i}} \cdot
  \frac{\partial}{\partial x^{i}} - 
  \frac{\partial}{\partial x^{0}} \cdot \frac{\partial}{\partial x^{0}}\\
  & = & \sum_{i=1}^{3}\left[ \sum_{j = 0}^{3} 
    \left(\frac{\partial y^{j}}{\partial x^{i}} \right)^{2}
    \frac{\partial}{\partial y^{j}} \cdot 
    \frac{\partial}{\partial y^{j}}\right] -
  \sum_{j=0}^{3} \left( \frac{\partial y^{j}}{\partial
      x^{0}} \right)^{2} 
  \frac{\partial}{\partial y^{j}} \cdot 
  \frac{\partial}{\partial y^{j}}\\
  & \stackrel{\gamma^{2} - \gamma^{2} \beta^{2} = 1}{=} & 
  l^{2} \left( \sum_{1}^{3} 
    \frac{\partial}{\partial y^{i}} \cdot 
    \frac{\partial}{\partial y^{i}} -
    \frac{\partial}{\partial y^{0}} \cdot 
    \frac{\partial}{\partial y^{0}} \right);
\end{eqnarray*}
allora $\Box = l^{2} \Box' $, da cui $\Box' = l^{-2} \Box$. Per quanto
riguarda $\varphi$ si ha $\Box' \varphi' = -\rho' $. Usando quanto
appena trovato e la~\eqref{eq:rhoprimo} con la~\eqref{eq:phiaerho}
otteniamo
\begin{eqnarray*}
  l^{-2} \Box \varphi' & = & 
  - \gamma l ^{-3} (1 - \beta u_{1}) \rho \Longrightarrow \\
  \Box \varphi ' & = & 
  \gamma l^{-1} (1- \beta u_{1}) \Box \varphi \\
  & = & \gamma l^{-1} (\Box \varphi - \beta \rho u_{1}) \\
  & = & \gamma l^{-1} \Box (\varphi - 
  \beta A_{1}) \Longrightarrow \\
  \varphi' & = & \gamma l^{-1} (\varphi - \beta A_{1}).
\end{eqnarray*}
Per $A_{1}'$ si ha $\Box ' A_{1}' = -\rho ' u_{1}'$; allora
\begin{eqnarray*}
  \Box A_{1}' & = & 
  \gamma l^{-1} \left (\rho u_{1} + \beta \rho \right) \\
  & = & \gamma l^{-1} 
  \left (\Box A_{1}  -\beta \Box \varphi \right),
\end{eqnarray*}
da cui $A_{1}' = \gamma l^{-1} (A_{1} - \beta \varphi)$. Analogamente
si procede per $A_{2}, A_{3}$, ottenendo
\begin{displaymath}
  A_{j}' = l ^{-1} A_{j}.
\end{displaymath}
Ora che si \`e capito come procedere, possiamo saltare tutti i
passaggi intermedi e riportare i risultati per il campo elettrico e
per il campo magnetico
\begin{displaymath}
  \begin{array}{rcl}
    E_{1}' & = & l^{-2} E_{1} \\
    E_{2}' & = & \gamma l^{-2} (E_{2} - \beta B_{3}) \\
    E_{3}' & = & \gamma l^{-2} (E_{3} + \beta B_{3})
  \end{array}
  \qquad
  \begin{array}{rcl}
    B_{1}' & = & l^{-2} B_{1} \\
    B_{2}' & = & \gamma l^{-2} (B_{2} + \beta E_{3}) \\
    B_{3}' & = & \gamma l^{-2} (B_{3} - \beta E_{2}). 
  \end{array}
\end{displaymath}
Le rimanenti dimostrazioni seguono da queste; infatti
la~\eqref{eq:phiea} segue da~\eqref{eq:phiaerho} e~\eqref{eq:rhoeu}, e
le equazioni~\eqref{eq:corrente},~\eqref{eq:rhoede} e~\eqref{eq:bede}
seguono da~\eqref{eq:phiaerho},~\eqref{eq:aephi} e~\eqref{eq:phiea}.

Per quanto riguarda la densit\`a di forza egli trova, correttamente
\begin{equation}
  f'_x = \gamma (f_x - \beta \bef{f \cdot v}) /l^5,
\label{eq:ftrax}
\end{equation}
\begin{equation}
  f'_{y,z} = f_{y,z}/l^5,
\label{eq:ftrayz}
\end{equation}
e in maniera analoga per la forza.

\section{Ulteriori sviluppi}
Nel suo lavoro, \poin{} non si ferma qua; infatti parlando del
principio di minima azione e delle trasformazioni di Lorentz, egli
scopre la relazione
\begin{equation}
  \label{eq:primoinv}
  l^4 [\varepsilon \bef{E}'^2 - \bef{B}'^2/\mu] = [\varepsilon
  \bef{E}^2 - \bef{B}^2/\mu] 
\end{equation}
che, essendo $l=1$, rappresenta un'invariante
dell'elettromagnetismo. Inoltre, parlando delle onde di Langevin,
ne trova un altro, e precisamente
\begin{equation}
  \label{eq:secondoinv}
  \bef{E'}\cdot\bef{B'} = \bef{E} \cdot \bef{B}.
\end{equation}

Nella parte dedicata alle trasformazioni di Lorentz e ai gruppi, egli
osserva come la forma quadratica
\begin{equation}
  \label{eq:ds}
  x^2 + y^2 + z^2 - t^2
\end{equation}
sia invariante sotto le trasformazioni di Lorentz (e questo permetter\`a
la definizione di una metrica sullo spazio).

Nella settima sezione, dedicata al moto quasi stazionario, \poin{} poi
trova che, per ogni forza, vale 
\begin{equation}
  \bef{F} = \frac{\de \bef{P}}{\de t}, 
   \mbox{ con } \bef{P} = m_0 v \gamma 
   \label{eq:forza_rel}
\end{equation}
ed \( m_0 \) = massa a riposo. 

Nell'ultima parte, la nona, egli osserva che con la posizione $ t\mapsto
t i $
\begin{citaz}
 le trasformazioni di Lorentz non sono altro che una rotazione dello
 spazio attorno all'origine, fissa.
\end{citaz} 

